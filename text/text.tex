% Do not forget to include Introduction
%---------------------------------------------------------------
%\chapter{Introduction}
% uncomment the following line to create an unnumbered chapter
\chapter*{Introduction}\addcontentsline{toc}{chapter}{Introduction}\markboth{Introduction}{Introduction}
%---------------------------------------------------------------
\setcounter{page}{1}

% The following environment can be used as a mini-introduction for a chapter. Use that any way it pleases you (or comment it out). It can contain, for instance, a summary of the chapter. Or, there can be a quotation.
%\begin{chapterabstract}
%\end{chapterabstract}

Modern high-performance network devices are usually proprietary systems that combine custom hardware, specialized operating systems, and tightly coupled software. 
While these solutions offer high throughput and re\-li\-a\-bi\-li\-ty, they are typically expensive, inflexible, and slower to evolve due to their closed design and development model.
Vector Packet Processing (VPP) is a high-performance network stack that operates at layers 2 to 4 of the ISO/OSI model. 
It was originally developed by Cisco Systems, Inc. (which is a world leader in networking) and open-sourced in 2016 under the Fast Data Project (FD.io), that is part of the Linux Foundation.
VPP brings the ability to perform efficient, high-speed packet processing on common off-the-shelf (COTS) hardware, across a wide range of platforms and operating systems.
Its open and flexible architecture opens the door to a new class of network applications that can be deployed and scaled more easily than traditional hardware appliances. 
In this way, VPP could represent a shift in the traditionally conservative networking world, echoing the ‘‘Mainframe to PC’’ revolution, 
where general-purpose systems replaced proprietary platforms, enabling broader innovation and accessibility.

Since VPP was open-sourced only recently, it has not yet been widely adopted by the market, and there are only a limited number of academic studies on the subject. As a result, this area remains underexplored. 
This thesis aims to contribute to this field by evaluating VPP's\footnote{The abbreviation VPP is also commonly used in academic literature to refer to a Virtual Power Plant.} performance, 
with a focus on comparing throughput, latency, and energy efficiency across different VPP configurations and a reference Linux-based router under various traffic conditions.
The goal is to understand the theoretical advantages of VPP, design and implement a suitable measurement infrastructure, define representative traffic scenarios, 
and ultimately evaluate whether VPP provides superior performance. The aim is also to characterize the specific conditions and configurations under which it outperforms the Linux-based reference system.
The results may support a recommendation for its deployment by small-to-mid-sized ISPs currently using Linux-based routers or planning to upgrade legacy systems.
Moreover, in light of the increasing focus on energy efficiency in IT infrastructure, the results may also inform ongoing discussions about sustainable network architecture.

With the development of AI and the growing demand for high-resolution streaming services, it is highly likely that the demand for internet bandwidth will continue to rise. 
This will result in an increased need for network equipment capable of processing larger volumes of data more efficiently. 
Therefore, it is crucial to explore technologies like VPP that are capable to handle this growing demand and to explore their energy efficiency.

This thesis is divided into two parts: Theoretical and Practical. 
The Theoretical part presents the traditional approach to networking and packet processing, as well as an overview of how VPP is designed and the principles on which it operates. 
Additionally, it introduces the testing scenarios that were used. 
The Practical part describes the testing infrastructure, presents the results of various measurements, and provides an analysis of the findings.

%---------------------------------------------------------------
\chapter{Theoretical part}
%---------------------------------------------------------------

%---------------------------------------------------------------
\section{Vector Packet Processing (VPP) and Its Operating Principles}
%---------------------------------------------------------------
\begin{chapterabstract}
This section describes the fundamental principles behind the Vector Packet Processing (VPP) technology, which aims to enable efficient and high-performance network packet processing. 
VPP is built on modern programming and architectural principles that allow maximum utilization of contemporary hardware, particularly in parallel processing and memory access optimization.
\end{chapterabstract}

The section begins with a brief description of traditional network traffic processing methods used by operating systems and their limitations in terms of performance and scalability. 
Following that, the architecture of VPP is explored in detail, explaining how packets are processed in vectors, the use of a node graph, 
and the various techniques that contribute to its high efficiency—such as I/O and compute batching, zero-copy methods, and lock-free multi-threading. 
The purpose of this section is to provide a theoretical foundation for understanding how VPP operates.

\subsection{Traditional network traffic processing}
A \textit{network packet} is a basic unit of data transmitted over a network. It consists of a \textit{header}, which includes control information such as source and destination IP addresses, 
and a \textit{payload}, which carries the actual user data. 
Packets are routed independently through the network and reassembled at the destination. 
This structure allows efficient and reliable communication, even over complex or unreliable network paths.

Currently, packet processing works as follows: a packet arrives at the network card, which then
issues a system call (syscall) to the operating system for packet processing. The microprocessor
must save the currently executing instruction, perform a context switch, locate the appropriate
service routine in the interrupt vector table, and handle the packet processing. Once completed, it
must restore the saved instruction, perform another context switch, and return to processing the
interrupted program.

This system for operating peripherals was designed under the assumption that the peripherals
would not request interrupts continuously, which is not the case with network devices that need
to process large volumes of data split into small parts. 
This method requires the microprocessor to execute a significant
number of instructions not directly related to packet processing. 
Chase et al. \cite{gallatin1999trapeze} discovered \footnote{kap. 3.3 obr. 6} that if MTU is 1500 bytes, then interrupt handling accounts for 20\% - 25\% of receiver packet-processing overhead.
Another disadvantage of tradidtional packet processing is the inefficient handling of cache memory; the processing of the packets one by one in response to
interrupts leads to frequent cache misses in both cache \& inctruction caches.\footnote{kap. 4.2}\cite{cox2000profiling} 

%-----------------------------------------------------
\subsection{An Introduction to VPP}

Vector Packet Processing (VPP) is a multi-platform network stack that operates at layers 2-4 of the ISO/OSI model and is developed by the FD.io project. 
It consists of a set of forwarding vertices arranged in an oriented graph and auxiliary software and provides out-of-the-box switch/router functionality.
Unlike traditional network stacks, which run in the kernel, VPP operates in user space.

In a traditional approach, packets are processed one by one. In contrast, VPP reads the largest available number of packets called vector from the network interface card (NIC) 
and processes the entire vector through a VPP node-graph one node at a time. Each node in this graph handles a specific part of the packet processing.
This approach reduces cache misses and spreads fixed overhead costs across multiple packets, lowering the average processing cost per packet. 
Additionally, it allows VPP to take advantage of multiple cores, enabling parallel processing, which significantly improves overall performance.

Vector Packet Processing (VPP) runs on common off-the-shelf hardware (COTS), ensuring its broad compatibility and flexibility for deployment. 
It supports various architectures such as x86, ARM, and Power, and can be deployed on both standard servers and embedded devices. 
The design of VPP is agnostic to hardware, kernel, and deployment platform, meaning it can operate across a wide range of systems, including bare metal servers, virtual machines (VMs), and containers. 
This approach allows VPP to be deployed on widely available infrastructure without the need for specialized hardware.\cite{fdio_what_is_vpp}

\subsection{Techniques used in VPP}

DOPAST !!! Low-level code optimization technique

According to Linguaglossa et al.\cite{LINGUAGLOSSA} VPP uses theese kernel-bypass techniques:

\begin{itemize}
  \item \textbf{Lock-Free Multi-Threading (LFMT)}
is a programming technique that leverages modern multi-core CPUs to increase system performance. In network applications, parallelism is achieved by running multiple threads in the same time. 
Ideally, the more threads are used, the better the system performance but only up to a saturation point beyond which additional threads bring no gainns. 
However, to reach this ideal performance, traditional synchronization mechanisms such as mutexes and semaphores must be avoided, as they introduce delays due to thread contention. 
Instead, lock-free architectures have to be used, allowing threads to operate independently without blocking each other. 
In the context of VPP this approach is enabled by hardware features like multi-queue NICs, 
which allow each thread to handle a distinct subset of traffic, ensuring efficient and parallel processing.

  \item \textbf{I/O batching (IOB)}
is a key technique used in VPP. 
Instead of raising an interrupt for every incoming packet, the network interface card (NIC) collects multiple packets into a buffer and triggers an interrupt only when the buffer is full. 
This reduces the overhead caused by frequent context switching and interrupt handling. 
VPP typically uses poll-mode drivers, which collect packets in batches without relying on interrupts. 
Moreover, the batching technique is applied system-wide in VPP. This approach maximizes CPU efficiency, improves cache usage, and delivers stable, high-throughput performance even under heavy load.

  \item \textbf{Compute batching (CB)} 
is a technique that extends I/O batching to the processing phase itself. 
Instead of processing one packet at a time, network functions are designed to operate on entire batches of packets. 
This approach minimizes overhead from function calls (such as context switches and stack setup) and improves instruction cache efficiency. 
When a batch of packets enters a processing function, only the first packet might cause an instruction cache miss, while the rest benefit from already-warmed cache.
Additionally it is possible to take advatage of instruction-level parallelism.
  
  \item \textbf{Receive-Side Scaling (RSS)}
is a hardware-based technique used by modern NICs to distribute incoming packets across multiple RX queues. 
This enables parallel packet processing by allowing each queue to be handled by a separate thread, improving scalability and throughput. 
Packet assignment is typically done using a hash function over packet header fields (e.g., the 5-tuple). 

  \item \textbf{Zero-Copy (Z-C)} 
is a technique used to eliminate unnecessary memory copying during packet processing. 
Instead of copying incoming packets from the network interface card (NIC) to a separate buffer via system calls, 
the NIC writes packets directly into a pre-allocated memory region that is shared with the user-space application via Direct Memory Access (DMA). 
This allows the application to access packet data without invoking system calls or duplicating memory, significantly reducing CPU overhead. 

  \item \textbf {Cache Coherence and Locality (CC\&L)} are critical factors in the performance of modern software-based packet processing systems. 
In current COTS architectures, memory access has become a major bottleneck, which is mitigated by a multi-level cache hierarchy.
Minimizing cache misses and maintaining data locality during packet processing is essential for achieving high performance and low latency.

  \item \textbf {Low-Level Parallelism (LLP)} 
refers to the ability to exploit the internal micro-architecture of modern CPUs, including multi-stage pipelines, arithmetic-logical units (ALUs), and branch predictors that help maintain pipeline efficiency. 
Well-optimized code can keep these pipelines full and execute multiple instructions per clock cycle, increasing overall throughput. 
Performance can be further improved by giving hints to the compiler -- such as indicating the likely outcome of conditional branches -- to reduce pipeline stalls. 
Vectorized packet processing and specific coding practices can take full advantage of these hardware features and VPP was specifically designed to taky advantage of LLP.

\end{itemize}

\subsection{VPP Processing Graph and Graph nodes}
At the core of VPP (Vector Packet Processing) lies the Packet Processing Graph, a directed graph composed of relatively small, modular \& loosely coupled nodes. 
Each node is designed to perform a specific task and there are 3 types of them: \textit{process}, \textit{input} \& \textit{internal}. 
Process nodes do not participate in the packet forwarding graph; instead, they handle timers, events, and other background tasks within the VPP runtime.
Input nodes are used for input of data and internal nodes are used for vector processing. Internal nodes also serves as output nodes. 
When a vector of packets is prepared by input node, it is then pushed through the internal nodes. 
During processing, the vector may be split if the batch contains packets of different protocols or types, as they may need to follow different paths through the graph
When the original vector is completely processed, the process repeats.
Illustration of this Processing Graph is shown in fig. \ref{fig:processing-graph}.

\begin{figure}[!htbp]
    \centering
    \includegraphics[width=0.7\linewidth]{images/processing-graph.jpg}
    \caption{Picture showing the VPP Processing Graph~\cite{LINGUAGLOSSA}}
    \label{fig:processing-graph}
\end{figure}

Thanks to VPP's modular design, the processing graph is highly customizable and extensible. 
New nodes -- referred to as plugins -- can be easily added to implement specific functionality or repleace existing ones. 
Plugins are shared libraries that are loaded during startup of VPP, and they are not dependent on the VPP source code, allowing them to be developed independently. 
Moreover, existing nodes can be rewired to modify the packet processing logic when necessary.\cite{LINGUAGLOSSA, DR:COMMAG-18, fdio_vpp_extensible_2021}






%---------------------------------------------------------------
\section{Implementation of Vector Packet Processing}
%---------------------------------------------------------------
%------------------------------------------
\subsection{VPP key architecture components}

VPP's dataplane is implemented by four main architectural layers: VPPINFRA, VNET, VLIB, and Plugins. 
Each layer provides distinct functionalities that support efficient networking operations, from low-level data structure management to high-level network function optimizations. 
The following sections describe these layers in detail: \footnote{https://my-vpp-docs.readthedocs.io/en/vpp-config/gettingstarted/developers/swarch/softwarearchitecture.html}

\begin{itemize}
  \item \textbf{VPPINFRA} -- layer providing foundational libraries for performing tasks with memory, vectors, rings, lookups in hash tables \& timers.
  \item \textbf{VNET} -- layer that deals with networking on layers 2 - 4 and is responsible for Control plane.
  \item \textbf{VLIB} -- layer that provides library for vector processing, implements CLI and handles application management functions.
  \item \textbf{Plugins} -- layer which is a set of plugins that allow for adding network functions and optimizations tailored to specific needs.
\end{itemize}

\subsubsection{VPPINFRA}
VPPINFRA is a collection of library services designed to offer high-performance capabilities for various tasks. 
It includes features such as dynamic arrays, hashes, bitmaps, high-precision real-time clock support, event logging, and data structure serialization. The following functionalities are implemented:

\begin{itemize}
  \item \textbf{Vectors} -- dynamically resized arrays with \textit{headers} defined by user. They serve as a core building block for other data structures (e.g., hash tables, pools) and allow efficient memory reuse via safe length resetting.
  \item \textbf{Bitmaps} -- dynamic bitmaps based on VPPINFRA vectors.
  \item \textbf{Pools} -- structures used to quickly allocate \& free fixed-size data structures. 
  \item \textbf{Hashes} -- structures thats provide fast key-value lookups, commonly mapping keys to indices in vectors or pools. Bihash is used in the data plane for fixed-size keys and is thread-safe, while the simpler hash is used in the control plane for exact string matching.
  \item \textbf{Timekeeping} -- service providing high-precision, low-cost timing based on CPU ticks. Since CPU ticks are not perfectly accurate, the system continuously adjusts its estimate of "ticks per second" by comparing with the kernel’s time. This results in precise and smooth time measurements without the need for expensive system calls. 
  \item \textbf{Timer wheel} -- system for efficiently managing timers or timeouts. It allows the user to define parameters like the number of wheels, slots per ring, and timers per object, optimizing time-based operations in systems requiring high-performance event management.
\end{itemize}

\subsubsection{VNET}
odmítám dělat teď

\subsubsection{VLIB}
Zítra je taky den

\subsubsection{Plugins}
Plugins are used to modify or create new features into the VPP. 
Developers can create plugins through a straightforward process, involving the generation of necessary files and integration into the system. 
After building, the new plugin can be loaded and tested within the VPP environment. 

VLIB supports a simple mechanism for loading and using plugins. 
VLIB client applications specify a directory where the plug-ins are located and can apply a filter to narrow down the search. 
Once the plug-ins are loaded, VLIB ensures they are correctly registered and ready for use.


%--------------------------------------
\subsection{Configuration and Startup}


%---------------------------------------------------------------
\section{Utilization of Vector Packet Processing}
%---------------------------------------------------------------
VPP supports a comprehensive set of Layer 2 to Layer 4 network functions. 
At Layer 2, it provides Ethernet bridging, MAC learning, VLAN tagging (including dot1q and QinQ), and support for L2 cross-connects and policers.

At Layer 3, VPP implements both IPv4 and IPv6 routing with ECMP support, NAT44/NAT64, and ACL-based filtering. 
It also supports tunneling mechanisms such as GTP-U, IP-in-IP, and VXLAN. Segment routing (SRv6), LISP, and punt redirect mechanisms are included as well.

At the transport layer (L4), basic UDP and TCP stack functionality is available. 

Additionally, supported features include PPPoE, the WireGuard VPN protocol, GRE tunneling, DHCP client and proxy functionality, and L2TPv3.\cite{fdio-vpp-features-2502}

According to the VPP's authors~\cite{fdio_what_is_vpp}, VPP can be for example effectively utilized as a virtual switch, virtual router, gateway or used as a basis for a firewall, IDS and load balancer.
It already includes enough features to be deployed in production environments.


\subsection{Integration with the SDN/NFV Ecosystem}
To meet the requirements of modern virtualized and cloud-native networking environments, Vector Packet Processing (VPP) was architected with a clear separation between the data plane and control plane. 
This design choice enables its integration into SDN and NFV frameworks, where packet forwarding logic can operate independently from centralized control mechanisms. 
VPP's modularity and userspace implementation allow it to function efficiently within dynamic, multi-tenant infrastructure, 
while remaining compatible with orchestration systems and control-plane protocols commonly used in such deployments

VPP is fully compatible with both Virtual Network Functions (VNFs) and Cloud-Native Network Functions (CNFs). 
Its modular architecture allows deployment in environments utilizing service function chaining, Kubernetes-based orchestration, or OpenStack-based infrastructures. 
Because of its userspace design and performance-optimized data plane, VPP can serve as the fast packet processing backend for SDN-controlled systems and NFV orchestrators.\cite{fdio2017whitepaper}

\subsection{VPP as a Complete Router Solution}
VPP is implemented solely as a data-plane, meaning it is not a complete routing solution on its own. 
VPP is dedicated to efficiently forwarding packets between interfaces based on routing rules and access control filters, 
but it does not include a native control-plane or support for dynamic routing protocols such as BGP or OSPF.

However, as demonstrated by the authors of the VBSR (VPP-Bird Software Router) project \cite{10819057}, 
it is possible to integrate VPP with additional components such as the Linux Control Plane (Linux-CP) plugin and the BIRD routing daemon. 
Bird acting as a control-plane enables dynamic routing using protocols like BGP 
and the Linux-CP is responsible for communication between VPP and BIRD. 
This integrated system creates a nearly feature-complete router solution, comparable in functionality to commercial routers.

It is important to note, however, that firewall functionality is still limited and was left by authors of VBSR as a future work.~\cite{10819057} 
While VPP supports basic packet filtering through ACLs, it lacks advanced stateful firewall features~\cite{fdio-vpp-features-2502}. These would need to be handled externally.



%---------------------------------------------------------------
\section{Survey of Traffic Generation Tools}
%---------------------------------------------------------------
Although numerous traffic generation tools exist \cite{traffic-generators}, 
this section focuses on a subset commonly used for high-performance benchmarking and synthetic traffic generation in research and practice, namely iPerf3, D-ITG, TRex, Pktgen-DPDK \& Genesids. 

\begin{itemize}
  \item \textbf{iPerf3} -- iPerf3 is a network testing tool used to measure TCP, UDP, and SCTP throughput between two endpoints. It allows detailed configuration of testing parameters such as buffer size, number of parallel streams, test duration, and jitter. iPerf3 can also measure jitter, providing insights into the variation in packet arrival times, which is useful for evaluating network stability. Its client-server architecture makes it a common tool for performance benchmarking of networks and devices.\cite{iperf}

  \item \textbf{TRex} -- TRex, developed by Cisco, is a high-performance, stateful and stateless traffic generator built on top of DPDK. It supports the generation of realistic Layer 4–7 traffic using pre-recorded PCAP files and emulates multiple concurrent users and flows. TRex is especially suited for benchmarking network function virtualization (NFV) platforms, routers, and firewalls in both laboratory and production-like environments.\cite{trex} 

\item \textbf{Pktgen-DPDK} -- Pktgen-DPDK is a high-performance traffic generator tool developed as part of the Data Plane Development Kit (DPDK). Pktgen-DPDK supports various network protocols, including IPv4, IPv6, UDP, and TCP. The tool allows precise control over traffic parameters, such as packet rate, size, and timing. Pktgen-DPDK is used in network performance tests and can capture packet-level statistics to assess the performance of the devices under test.\cite{pktgen_dpdk} 
\end{itemize}





%---------------------------------------------------------------
\chapter{Pratical part}
%---------------------------------------------------------------

%---------------------------------------------------------------
\section{Building Infrastructure for Measurement}
%---------------------------------------------------------------
The testing infrastructure has been implemented as recomended in RFC 2544~\cite{rfc2544}, which defines methods for evaluating network performance. 
It consists of a device under test (DUT), connected to a measurement device called Tester\footnote{The hardware used in this testing setup was loaned free of charge for the purposes of this bachelor thesis by Silicon Hill club.}.

The Device Under Test (DUT) and the measurement device are connected using 100~Gbit capable cables, preventing any potential bottlenecks in the connection.
The illustration of this hardware setup is shown in fig. \ref{fig:hardware-setup}

\begin{figure}[!htbp]
    \centering
    \includegraphics[width=0.9\linewidth]{images/setup.png}
    \caption{Picture showing hardware setup}
    \label{fig:hardware-setup}
\end{figure}

The Device Under Test (DUT) is the network device being evaluated during testing. 
It is configured with a specific network stack and settings based on measurement scenario 
and serves as the focus of performance and behavior analysis in a controlled test environment. 
The DUT is responsible for processing network traffic and responding to the test conditions set by the measurement device.
Additionally, the electrical power consumption of the DUT is monitored and measured during the tests to assess its energy efficiency under varying loads.
The hardware of DUT is shown in table \ref{tab:hardware_dut}.

\begin{table}[h!]
\centering
\caption{Hardware details for DUT (Device Under Test)}
\begin{tabular}{|c|c|}
\hline
\textbf{Hardware Component} & \textbf{DUT (Device Under Test)} \\
\hline
\textbf{CPU Model} & 2x Intel(R) Xeon(R) CPU E5-2660 v3 \\
\hline
\textbf{Frequency} & 2.60GHz \\
\hline
\textbf{Cores} & 10 physical cores each (one thread per core) \\
\hline
\textbf{Memory (RAM)} & TODO! Size, type, speed \\
\hline
\textbf{Network Interface Cards (NIC)} & Mellanox ConnectX-6 Dx (Dual-port) \\
\hline
\end{tabular}
\label{tab:hardware_dut}
\end{table}

The Tester (Measurement Device), on the other hand, is responsible in generating the network traffic and capturing the responses from the DUT.
Its physical features are shown in table \ref{tab:hardware_tester}. 

\begin{table}[h!]
\centering
\caption{Hardware details for Tester (Measurement Device)}
\begin{tabular}{|c|c|}
\hline
\textbf{Hardware Component} & \textbf{Tester (Measurement Device)} \\
\hline
\textbf{CPU Model} & 2x Intel(R) Xeon(R) Gold 6136 CPU \\
\hline
\textbf{Frequency} & 3.00GHz \\
\hline
\textbf{Cores} & 12 physical cores each (two threads per core)\\
\hline
\textbf{Memory (RAM)} & TODO! Size, type, speed \\
\hline
\textbf{Network Interface Cards (NIC)} & 2x Mellanox ConnectX-5 \\
\hline
\end{tabular}
\label{tab:hardware_tester}
\end{table}

The DUT is running Debian GNU/Linux 12 (Bookworm) x86\_64 with Linux kernel version \textit{6.1.0-32-amd64}, VPP v25.02-release, and DPDK version 24.11.1. 
This kernel version is the current standard long-term support (LTS) release provided with Debian 12 (Bookworm) and was used for all tests involving the Linux networking stack.

The tester generates traffic using Cisco TRex version 3.06.


%---------------------------------------------------------------
\section{Metodology}
%---------------------------------------------------------------
The RFC 2544 recommends to test be at least 60 seconds in duration\cite{rfc2544} and NAJÍT ZDROJ??? kolikrát opakovat
Each test scenario executed using TRex was repeated 30 times, with each individual run lasting five minutes. 
The reported results represent the arithmetic mean of these 30 measurements. 
In cases where an anomalous spike or irregularity was observed in the results, the corresponding measurement was discarded and the test was repeated.  
All this steps shoudl ensure consistency and statistical reliability.

Transmitted packets and bytes refer exclusively to those that were successfully sent and received without any loss (i.e., without packet drops).

To evaluate energy efficiency, the number of packets per watt (PPW) and bytes per watt (BPW) was used, with all values rounded to two decimal places. 
These metrics provide a measure of how efficiently energy is utilized for each transmitted packet and byte. 
The machine -- when idle -- consumes 144~Watts per minute.


%---------------------------------------------------------------
\section{Test Scenarios and Results}
%---------------------------------------------------------------
To provide a comprehensive and representative view, the tests are structured into five subsections, each corresponding to a different Ethernet frame size. 
Four of the selected sizes -- 64 bytes, 512 bytes, 1280 bytes, and 1518 bytes -- are recommended by RFC~2544~\cite{rfc2544} 
covering both edge cases and practically relevant intermediate values. 
The fifth size, 889 bytes, was chosen based on real-world traffic analysis by Jurkiewicz et al.~\cite{JURKIEWICZ202115}, who identified it as the average frame size observed in modern network environment.
This selection covers the full range of standard Ethernet frame sizes, from the minimum to the maximum non-jumbo frames, 
while also including a statistically representative average.

All tests were conducted at four different transmission speeds -- 1, 10, 25, and 40 Gbit/s -- to evaluate the behavior of each configuration under varying network loads.

Traffic in all scenarios is generated using TRex with the TBD profile,
which ensures that each packet carries a unique source IP address to simulate multiple concurrent clients, while maintaining a single destination IP per direction.
Since the aim of this thesis is to evaluate the VPP architecture rather than specific features (e.g., routing table lookup or hashing mechanisms),
the routing table of the DUT contains only two active forwarding entries corresponding to the test routes,
along with two administrative entries used for management purposes.

The DUT is configured with the VPP stack and tested under three levels of parallelism: using 1, 4, and 10 worker threads, plus a single main thread in all configurations.
The worker threads are pinned to the NUMA node closest to the NICs to minimize memory access latency.
The number of RX/TX queues is aligned with the number of active worker threads in each configuration to ensure balanced packet distribution and optimal resource utilization.
In the tables, the number of worker threads is denoted as VPP-X, where X indicates the number of worker threads used.

To provide a baseline for comparison, all scenarios are also executed using the standard Linux kernel networking stack.
It is configured with routing and interface parameters equivalent to the VPP setup, utilizing all 10 CPU cores on the NUMA node closest to the NICs.
RPS is enabled, with affinities set evenly across the cores.
This allows for a direct comparison between VPP and traditional kernel-based forwarding in terms of performance and energy efficiency.

\subsection{One-way forwarding}

These tests were conducted in a one-way configuration, as suggested by RFC~2544~\cite{rfc2544}.
This scenario can simulate networks with asymmetric traffic patterns (e.g., a~web server) or reflect conditions similar to a DoS attack.

%----------------------------------
\subsubsection{1 Gbps Test Results}

\begin{table}[h!]
\centering
\caption{Results of one-way 1~Gbit/s tests}
\begin{tabular}{|c|l|r|r|r|r|}
\hline
\textbf{} & \textbf{Config} & \textbf{Energy [Wh]} & \textbf{Pkt Loss [\%]} & \textbf{Avg Lat [$\mu$s]} & \textbf{Jitter [$\mu$s]} \\
\hline
\multirow{4}{*}{\rotatebox{90}{64B}}    & VPP-1  & 5.68 & 0.00 & 12.8   & 9.5   \\
                                        & VPP-4  & 6.46 & 0.00 & 27.45  & 13.55 \\
                                        & VPP-10 & 7.86 & 0.00 & 28.3   & 12.65 \\
                                        & Linux  & 6.78 & 0.00 & 108.05 & 97.25 \\
\hline
\multirow{4}{*}{\rotatebox{90}{512B}}   & VPP-1  & 5.69 & 0.00 & 12.1   & 11.7  \\
                                        & VPP-4  & 6.48 & 0.00 & 22.85  & 17.4  \\
                                        & VPP-10 & 7.91 & 0.00 & 23.80  & 17.35 \\
                                        & Linux  & 6.23 & 0.00 & 56.1   & 51.35 \\
\hline
\multirow{4}{*}{\rotatebox{90}{889B}}   & VPP-1  & 5.76 & 0.00 & 10.3   & 8.4   \\
                                        & VPP-4  & 6.45 & 0.00 & 21.3   & 17.6  \\
                                        & VPP-10 & 7.85 & 0.00 & 20.95  & 16.3  \\
                                        & Linux  & 6.14 & 0.00 & 23.70  & 21.5  \\
\hline
\multirow{4}{*}{\rotatebox{90}{1280B}}  & VPP-1  & 5.72 & 0.00 & 8.1    & 5.8   \\
                                        & VPP-4  & 6.46 & 0.00 & 18.3   & 17.2  \\
                                        & VPP-10 & 7.84 & 0.00 & 18.8   & 15.95 \\
                                        & Linux  & 6.11 & 0.00 & 9.95   & 3.80  \\
\hline
\multirow{4}{*}{\rotatebox{90}{1518B}}  & VPP-1  & 5.66 & 0.00 & 7.2    & 5.5   \\
                                        & VPP-4  & 6.42 & 0.00 & 16.1   & 15.75 \\
                                        & VPP-10 & 7.84 & 0.00 & 18.8   & 15.95 \\
                                        & Linux  & 6.10 & 0.00 & 12.25  & 10.55 \\
\hline
\end{tabular}
\label{tab:1gbit_final_clean}
\end{table}




%---------------------------------------------------------------------------------------------
%---------------------------------------------------------------------------------------------
%---------------------------------------------------------------------------------------------
\subsubsection{10 Gbps Test Results}

\begin{table}[h!]
\centering
\caption{Results of one-way 10~Gbit/s tests}
\begin{tabular}{|c|l|r|r|r|r|}
\hline
\textbf{} & \textbf{Config} & \textbf{Energy [Wh]} & \textbf{Pkt Loss [\%]} & \textbf{Avg Lat [$\mu$s]} & \textbf{Jitter [$\mu$s]} \\
\hline
\multirow{4}{*}{\rotatebox{90}{64B}}    & VPP-1  & 5.72 & 59.50 & 589.50  & 14.05  \\
                                        & VPP-4  & 6.56 & 0.02  & 20.60   & 6.00   \\
                                        & VPP-10 & 8.04 & 0.00  & 30.00   & 12.90  \\
                                        & Linux  & 7.35 & 80.97 & 3846.30 & 217.80 \\
\hline
\multirow{4}{*}{\rotatebox{90}{512B}}   & VPP-1  & 5.60 & 0.00  & 19.95   & 12.00  \\
                                        & VPP-4  & 6.48 & 0.00  & 28.20   & 14.50  \\
                                        & VPP-10 & 7.97 & 0.00  & 29.05   & 13.65  \\
                                        & Linux  & 6.85 & 0.00  & 129.40  & 99.35  \\
\hline
\multirow{4}{*}{\rotatebox{90}{889B}}   & VPP-1  & 5.62 & 0.00  & 23.40   & 14.95  \\
                                        & VPP-4  & 6.48 & 0.00  & 26.50   & 18.25  \\
                                        & VPP-10 & 7.95 & 0.00  & 26.95   & 17.55  \\
                                        & Linux  & 6.67 & 0.00  & 66.45   & 68.30  \\
\hline
\multirow{4}{*}{\rotatebox{90}{1280B}}  & VPP-1  & 5.58 & 0.00  & 22.80   & 17.05  \\
                                        & VPP-4  & 6.47 & 0.00  & 25.95   & 16.90  \\
                                        & VPP-10 & 7.98 & 0.00  & 26.80   & 16.60  \\
                                        & Linux  & 6.54 & 0.00  & 57.05   & 65.50  \\
\hline
\multirow{4}{*}{\rotatebox{90}{1518B}}  & VPP-1  & 5.57 & 0.00  & 20.45   & 14.95  \\
                                        & VPP-4  & 6.48 & 0.00  & 24.00   & 17.45  \\
                                        & VPP-10 & 7.97 & 0.00  & 25.40   & 17.70  \\
                                        & Linux  & 6.51 & 0.00  & 61.75   & 65.80  \\
\hline
\end{tabular}
\label{tab:10gbit_combined}
\end{table}



Figure~\ref{fig:10g} shows the energy efficiency of each configuration in this test in terms of delivered packets and bytes.
The significant drop in performance for VPP-1 and Linux in 64-byte frames test is caused by large packet loss.
When all packets are successfully delivered, all VPP configurations maintain stable BPWh values, which is due to their busy-wait processing model.
The Linux stack, on the other hand, becomes more efficient with increasing frame size, likely as a result of less frequent system calls.

\begin{figure}[!htbp]
    \centering
    \includegraphics[width=\linewidth]{images/consumption-10g.png}
    \caption{Energy efficiency per delivered data in one-way 10\,Gbit/s.}
    \label{fig:10g}
\end{figure}



%---------------------------------------------------------------------------------------------
%---------------------------------------------------------------------------------------------
%---------------------------------------------------------------------------------------------
\subsubsection{25 Gbps Test Results}

\begin{table}[h!]
\centering
\caption{Results of one-way 25~Gbit/s tests}
\begin{tabular}{|c|l|r|r|r|r|}
\hline
\textbf{} & \textbf{Config} & \textbf{Energy [Wh]} & \textbf{Pkt Loss [\%]} & \textbf{Avg Lat [$\mu$s]} & \textbf{Jitter [$\mu$s]} \\
\hline
\multirow{4}{*}{\rotatebox{90}{64B}}    & VPP-1  & 5.59 & 83.29 & 577.50   & 7.70   \\
                                        & VPP-4  & 6.56 & 49.00 & 198.85   & 7.65   \\
                                        & VPP-10 & 8.26 & 29.55 & 155.35   & 11.95  \\
                                        & Linux  & 7.43 & 92.71 & 5597.95  & 632.00 \\
\hline
\multirow{4}{*}{\rotatebox{90}{512B}}   & VPP-1  & 5.57 & 0.07  & 31.85    & 11.15  \\
                                        & VPP-4  & 6.43 & 0.00  & 29.55    & 13.75  \\
                                        & VPP-10 & 8.03 & 0.00  & 31.00    & 14.45  \\
                                        & Linux  & 7.57 & 47.97 & 7819.60  & 477.55 \\
\hline
\multirow{4}{*}{\rotatebox{90}{889B}}   & VPP-1  & 5.67 & 0.00  & 23.45    & 11.15  \\
                                        & VPP-4  & 6.39 & 0.00  & 30.50    & 15.50  \\
                                        & VPP-10 & 8.01 & 0.00  & 29.45    & 15.55  \\
                                        & Linux  & 7.42 & 0.87  & 166.05   & 111.15 \\
\hline
\multirow{4}{*}{\rotatebox{90}{1280B}}  & VPP-1  & 5.68 & 0.00  & 23.85    & 10.30  \\
                                        & VPP-4  & 6.48 & 0.00  & 31.15    & 15.40  \\
                                        & VPP-10 & 7.98 & 0.00  & 28.40    & 16.15  \\
                                        & Linux  & 7.10 & 0.00  & 146.90   & 126.20 \\
\hline
\multirow{4}{*}{\rotatebox{90}{1518B}}  & VPP-1  & 5.70 & 0.00  & 24.45    & 14.90  \\
                                        & VPP-4  & 6.46 & 0.00  & 30.60    & 15.80  \\
                                        & VPP-10 & 7.96 & 0.00  & 28.90    & 15.65  \\
                                        & Linux  & 7.00 & 0.00  & 130.00   & 105.30 \\
\hline
\end{tabular}
\label{tab:25gbit_combined}
\end{table}



Figure~\ref{fig:25g} illustrates the energy efficiency of each configuration in terms of delivered packets and bytes.
Compared to the previous test, the Linux stack performed significantly worse, while VPP maintained stable BPWh, except in the case of 64-byte frames.

\begin{figure}[!htbp]
    \centering
    \includegraphics[width=\linewidth]{images/consumption-25g.png}
    \caption{Energy efficiency per delivered data in one-way 25\,Gbit/s.}
    \label{fig:25g}
\end{figure}


%-----------------------------------
\subsubsection{40 Gbps Test Results}

\begin{table}[h!]
\centering
\caption{Results of one-way 40~Gbit/s tests}
\begin{tabular}{|c|l|r|r|r|r|}
\hline
\textbf{} & \textbf{Config} & \textbf{Energy [Wh]} & \textbf{Pkt Loss [\%]} & \textbf{Avg Lat [$\mu$s]} & \textbf{Jitter [$\mu$s]} \\
\hline
\multirow{4}{*}{\rotatebox{90}{64B}}    & VPP-1  & 5.60 & 89.53 & 576.00   & 6.50   \\
                                        & VPP-4  & 6.57 & 67.54 & 195.70   & 6.57   \\
                                        & VPP-10 & 8.12 & 54.38 & 152.05   & 9.20   \\
                                        & Linux  & 7.34 & 95.43 & 5629.05  & 550.30 \\
\hline
\multirow{4}{*}{\rotatebox{90}{512B}}   & VPP-1  & 5.82 & 35.26 & 292.45   & 122.15 \\
                                        & VPP-4  & 6.56 & 0.00  & 28.80    & 9.90   \\
                                        & VPP-10 & 8.00 & 0.00  & 35.45    & 14.35  \\
                                        & Linux  & 7.56 & 69.84 & 6621.50  & 892.35 \\
\hline
\multirow{4}{*}{\rotatebox{90}{889B}}   & VPP-1  & 5.77 & 3.85  & 203.05   & 25.20  \\
                                        & VPP-4  & 6.54 & 0.00  & 32.55    & 13.75  \\
                                        & VPP-10 & 8.00 & 0.00  & 33.60    & 18.50  \\
                                        & Linux  & 7.63 & 49.85 & 7222.80  & 315.40 \\
\hline
\multirow{4}{*}{\rotatebox{90}{1280B}}  & VPP-1  & 5.79 & 0.00  & 32.50    & 14.65  \\
                                        & VPP-4  & 6.40 & 0.00  & 33.95    & 14.35  \\
                                        & VPP-10 & 8.02 & 0.00  & 32.15    & 16.85  \\
                                        & Linux  & 7.59 & 6.25  & 2830.95  & 104.10 \\
\hline
\multirow{4}{*}{\rotatebox{90}{1518B}}  & VPP-1  & 5.83 & 0.00  & 30.35    & 15.75  \\
                                        & VPP-4  & 6.43 & 0.00  & 34.20    & 14.75  \\
                                        & VPP-10 & 8.02 & 0.00  & 33.30    & 14.80  \\
                                        & Linux  & 7.36 & 0.08  & 195.05   & 122.05 \\
\hline
\end{tabular}
\label{tab:40gbit_combined}
\end{table}



\subsection{Bidirectional forwarding}

In accordance with the recommendations of RFC~8219~\cite{RFC8219}, which -- although focused on IPv6 transition technologies -- 
also defines general benchmarking principles, the same test scenarios were executed under bidirectional traffic conditions.  
This configuration reflects a more realistic networking environment where traffic flows simultaneously in both directions, such as in point-to-point communications,a
 VPN tunnels, or client-server interactions involving both requests and responses.  
Unlike the one-way setup, bidirectional forwarding places a greater strain on system resources by utilizing both receive (RX) and transmit (TX) paths concurrently, 
which may reveal bottlenecks or limitations not evident in unidirectional traffic scenarios.

\subsubsection{1 Gbps Test Results}


\begin{table}[h!]
\centering
\caption{Results of bidirectional 1~Gbit/s tests}
\begin{tabular}{|c|l|r|r|r|r|}
\hline
\textbf{} & \textbf{Config} & \textbf{Energy [Wh]} & \textbf{Pkt Loss [\%]} & \textbf{Avg Lat [$\mu$s]} & \textbf{Jitter [$\mu$s]} \\
\hline
\multirow{4}{*}{\rotatebox{90}{64B}}    
    & VPP-1  & 5.42 & 0.00 & 27.83 & 10.48 \\
    & VPP-4  & 6.30 & 0.00 & 30.18 & 14.08 \\
    & VPP-10 & 8.00 & 0.00 & 30.45 & 13.48 \\
    & Linux  & 7.20 & 0.00 & 147.68 & 100.05 \\
\hline
\multirow{4}{*}{\rotatebox{90}{512B}}   
    & VPP-1  & 5.80 & 0.00 & 28.05 & 15.70 \\
    & VPP-4  & 6.45 & 0.00 & 26.73 & 17.30 \\
    & VPP-10 & 7.93 & 0.00 & 24.56 & 16.30 \\
    & Linux  & 6.35 & 0.00 & 39.83 & 39.20 \\
\hline
\multirow{4}{*}{\rotatebox{90}{889B}}   
    & VPP-1  & 5.66 & 0.00 & 26.10 & 14.50 \\
    & VPP-4  & 6.43 & 0.00 & 26.13 & 17.53 \\
    & VPP-10 & 7.78 & 0.00 & 26.28 & 17.00 \\
    & Linux  & 6.27 & 0.00 & 17.4  & 12.38 \\
\hline
\multirow{4}{*}{\rotatebox{90}{1280B}}  
    & VPP-1  & 5.75 & 0.00 & 25.20 & 12.45 \\
    & VPP-4  & 6.39 & 0.00 & 25.23 & 17.10 \\
    & VPP-10 & 7.85 & 0.00 & 27.40 & 18.33 \\
    & Linux  & 6.21 & 0.00 & 12.68 & 7.28  \\
\hline
\multirow{4}{*}{\rotatebox{90}{1518B}}  
    & VPP-1  & 5.77 & 0.00 & 25.10 & 13.15 \\
    & VPP-4  & 6.36 & 0.00 & 25.53 & 15.63 \\
    & VPP-10 & TODO?&      &       &       \\
    & Linux  & 6.13 & 0.00 & 12.40  & 9.40  \\
\hline
\end{tabular}
\label{tab:bidirectional-1g}
\end{table}


%-----------------------------------
%-----------------------------------
%-----------------------------------
\subsubsection{10 Gbps Test Results}


\begin{table}[h!]
\centering
\caption{Results of bidirectional 10~Gbit/s tests}
\begin{tabular}{|c|l|r|r|r|r|}
\hline
\textbf{} & \textbf{Config} & \textbf{Energy [Wh]} & \textbf{Pkt Loss [\%]} & \textbf{Avg Lat [$\mu$s]} & \textbf{Jitter [$\mu$s]} \\
\hline
\multirow{4}{*}{\rotatebox{90}{64B}}    
    & VPP-1  & 5.52 & 70.40 & 537.43 & 12.15 \\
    & VPP-4  & 6.51 & 30.47 & 188.40 & 7.23  \\
    & VPP-10 & 8.40 & 0.00  & 42.05  & 12.55 \\
    & Linux  & 7.24 & 92.91 & 8408.15 & 924.05 \\
\hline
\multirow{4}{*}{\rotatebox{90}{512B}}   
    & VPP-1  & 5.80 & 0.00 & 27.75 & 11.73 \\
    & VPP-4  & 6.55 & 0.00 & 32.50 & 14.48 \\
    & VPP-10 & 7.99 & 0.00 & 30.58 & 15.38 \\
    & Linux  & 7.44 & 25.22 & 3837.13 & 172.03 \\
\hline
\multirow{4}{*}{\rotatebox{90}{889B}}   
    & VPP-1  & 5.77 & 0.00 & 34.00 & 15.15 \\
    & VPP-4  & 6.55 & 0.00 & 32.50 & 14.48 \\
    & VPP-10 & 7.98 & 0.00 & 30.50 & 16.35 \\
    & Linux  & 7.19 & 0.00 & 110.20 & 82   \\
\hline
\multirow{4}{*}{\rotatebox{90}{1280B}}  
    & VPP-1  & 5.81 & 0.00 & 34.15 & 14.86 \\
    & VPP-4  & 6.37 & 0.00 & 32.73 & 16.88 \\
    & VPP-10 & 8.01 & 0.00 & 32.20 & 15.85 \\
    & Linux  & 6.97 & 0.00 & 77.45 & 70.33 \\
\hline
\multirow{4}{*}{\rotatebox{90}{1518B}}  
    & VPP-1  & 5.78 & 0.00 & 33.80 & 16.10 \\
    & VPP-4  & 6.36 & 0.00 & 25.56 & 15.63 \\
    & VPP-10 & 7.85 & 0.00 & 27.40 & 18.33 \\
    & Linux  & 6.89 & 0.00 & 72.78 & 70.08 \\
\hline
\end{tabular}
\label{tab:bidirectional-10g}
\end{table}

%-----------------------------------
%-----------------------------------
%-----------------------------------
\subsubsection{25 Gbps Test Results}

\begin{table}[h!]
\centering
\caption{Results of bidirectional 25~Gbit/s tests}
\begin{tabular}{|c|l|r|r|r|r|}
\hline
\textbf{} & \textbf{Config} & \textbf{Energy [Wh]} & \textbf{Pkt Loss [\%]} & \textbf{Avg Lat [$\mu$s]} & \textbf{Jitter [$\mu$s]} \\
\hline
\multirow{4}{*}{\rotatebox{90}{64B}}    
    & VPP-1  & 5.54 & 88.16 & 545.50 & 12.30 \\
    & VPP-4  & 6.60 & 64.58 & 185.08 & 8.18  \\
    & VPP-10 & 8.40 & 42.05 & 450.63 & 12.55 \\
    & Linux  & 7.23 & 97.09 & 9266.88 & 1409.90 \\
\hline
\multirow{4}{*}{\rotatebox{90}{512B}}   
    & VPP-1  & 5.80 & 28.95 & 294.13 & 25.80 \\
    & VPP-4  & 6.72 & 0.00  & 38.18  & 13.28 \\
    & VPP-10 & 8.20 & 0.00  & 37.38  & 15.43 \\
    & Linux  & 7.53 & 76.86 & 8079.28 & 644.00 \\
\hline
\multirow{4}{*}{\rotatebox{90}{889B}}   
    & VPP-1  & 5.82 & 0.01  & 43.63 & 17.15 \\
    & VPP-4  & 6.71 & 0.00  & 39.58  & 16.38 \\
    & VPP-10 & 8.16 & 0.00  & 36.83  & 16.65 \\
    & Linux  & 7.50 & 57.50 & 16506.75 & 814.28 \\
\hline
\multirow{4}{*}{\rotatebox{90}{1280B}}  
    & VPP-1  & 5.86 & 0.00  & 40.40  & 15.73 \\
    & VPP-4  & 6.63 & 0.00  & 39.43  & 15.55 \\
    & VPP-10 & 8.22 & 0.00  & 37.93  & 15.25 \\
    & Linux  & 7.50 & 32.67 & 13791.63 & 357.95 \\
\hline
\multirow{4}{*}{\rotatebox{90}{1518B}}  
    & VPP-1  & 5.84 & 0.00  & 43.95  & 18.58 \\
    & VPP-4  & 6.65 & 0.00  & 38.38  & 15.88 \\
    & VPP-10 & 8.17 & 0.00  & 36.28  & 17.78 \\
    & Linux  & 7.70 & 10.57 & 6568.30 & 112.68 \\
\hline
\end{tabular}
\label{tab:bidirectional-25g}
\end{table}







\subsubsection{40 Gbps Test Results}

\begin{table}[h!]
\centering
\caption{Results of bidirectional 40~Gbit/s tests}
\begin{tabular}{|c|l|r|r|r|r|}
\hline
\textbf{} & \textbf{Config} & \textbf{Energy [Wh]} & \textbf{Pkt Loss [\%]} & \textbf{Avg Lat [$\mu$s]} & \textbf{Jitter [$\mu$s]} \\
\hline
\multirow{4}{*}{\rotatebox{90}{64B}}    
    & VPP-1  & 5.52 & 92.60 & 543.45 & 13.25 \\
    & VPP-4  & 6.64 & 77.42 & 182.70 & 6.58  \\
    & VPP-10 & 8.38 & 71.13 & 156.08 & 10.375 \\
    & Linux  & 7.22 & 98.22 & 9256.2 & 1353.3 \\
\hline
\multirow{4}{*}{\rotatebox{90}{512B}}   
    & VPP-1  & 5.82 & 57.83 & 317.70 & 103.95 \\
    & VPP-4  & 6.74 & 19.06 & 175.25 & 21.63 \\
    & VPP-10 & 8.42 & 13.00 & 168.70 & 20.28 \\
    & Linux  & 7.59 & 86.28 & 8976.75 & 924.63 \\
\hline
\multirow{4}{*}{\rotatebox{90}{889B}}   
    & VPP-1  & 5.84 & 37.21 & 215.45 & 47.30 \\
    & VPP-4  & 6.89 & 1.29  & 142.7  & 17.2 \\
    & VPP-10 & 8.43 & 0.00  & 52.48  & 15.38 \\
    & Linux  & 7.64 & 76.43 & 17567.73 & 1555.53 \\
\hline
\multirow{4}{*}{\rotatebox{90}{1280B}}  
    & VPP-1  & 5.89 & 18.32 & 179.13 & 36.65 \\
    & VPP-4  & 6.86 & 0.00  & 54.68  & 18.28 \\
    & VPP-10 & 8.39 & 0.00  & 49.63  & 17.40 \\
    & Linux  & 7.48 & 64.37 & 16067.48 & 673.3 \\
\hline
\multirow{4}{*}{\rotatebox{90}{1518B}}  
    & VPP-1  & 5.86 & 14.67 & 159.55 & 36.45 \\
    & VPP-4  & 6.86 & 0.00  & 57.20  & 20.43 \\
    & VPP-10 & 8.38 & 0.00  & 51.63  & 19.63 \\
    & Linux  & 7.56 & 54.60 & 12835.1 & 588.1 \\
\hline
\end{tabular}
\label{tab:bidirectional-40g}
\end{table}



\subsection{NAT}

%----------------------------------
%----------------------------------
%----------------------------------
\subsubsection{1 Gbps Test Results}

\begin{table}[h!]
\centering
\caption{Results of NAT 1~Gbit/s tests}
\begin{tabular}{|c|l|r|r|r|r|}
\hline
\textbf{} & \textbf{Config} & \textbf{Energy [Wh]} & \textbf{Pkt Loss [\%]} & \textbf{Avg Lat [$\mu$s]} & \textbf{Jitter [$\mu$s]} \\
\hline
\multirow{4}{*}{\rotatebox{90}{64B}} &
          VPP-1  & 5.67  &  1.56 & 19.10 & 11.65 \\
        & VPP-4  &       &       &       &       \\
        & VPP-10 &       &       &       &       \\
        & Linux  &       &       &       &       \\
\hline
\multirow{4}{*}{\rotatebox{90}{512B}} &
          VPP-1  & 5.76  &  1.53 & 12.25 & 12.25 \\
        & VPP-4  &       &       &       &       \\
        & VPP-10 &       &       &       &       \\
        & Linux  &       &       &       &       \\
\hline
\multirow{4}{*}{\rotatebox{90}{889B}} &
          VPP-1  & 5.74  & 1.50  & 10.3  & 7.45  \\
        & VPP-4  &       &       &       &       \\
        & VPP-10 &       &       &       &       \\
        & Linux  &       &       &       &       \\
\hline
\multirow{4}{*}{\rotatebox{90}{1280B}} &
          VPP-1  & 5.73  & 1.48  & 7.70  &  7.30 \\
        & VPP-4  &       &       &       &       \\
        & VPP-10 &       &       &       &       \\
        & Linux  &       &       &       &       \\
\hline
\multirow{4}{*}{\rotatebox{90}{1518B}} &
          VPP-1  & 5.68  & 1.47  & 7.3   & 5.55  \\
        & VPP-4  &       &       &       &       \\
        & VPP-10 &       &       &       &       \\
        & Linux  &       &       &       &       \\
\hline
\end{tabular}
\label{tab:nat-1g}
\end{table}






















%----------------------------------
%----------------------------------
%----------------------------------
\subsubsection{10 Gbps Test Results}

\begin{table}[h!]
\centering
\caption{Results of NAT 10~Gbit/s tests}
\begin{tabular}{|c|l|r|r|r|r|}
\hline
\textbf{} & \textbf{Config} & \textbf{Energy [Wh]} & \textbf{Pkt Loss [\%]} & \textbf{Avg Lat [$\mu$s]} & \textbf{Jitter [$\mu$s]} \\
\hline
\multirow{4}{*}{\rotatebox{90}{64B}} &
          VPP-1  & 5.75  & 75.12 & 259.15& 11.95 \\
        & VPP-4  &       &       &       &       \\
        & VPP-10 &       &       &       &       \\
        & Linux  &       &       &       &       \\
\hline
\multirow{4}{*}{\rotatebox{90}{512B}} &
          VPP-1  & 5.56  & 1.56  & 31.15 & 16.9  \\
        & VPP-4  &       &       &       &       \\
        & VPP-10 &       &       &       &       \\
        & Linux  &       &       &       &       \\
\hline
\multirow{4}{*}{\rotatebox{90}{889B}} &
          VPP-1  & 5.57  & 1.56  & 29.8  & 18.55 \\
        & VPP-4  &       &       &       &       \\
        & VPP-10 &       &       &       &       \\
        & Linux  &       &       &       &       \\
\hline
\multirow{4}{*}{\rotatebox{90}{1280B}} &
          VPP-1  & 5.66  & 1.55  & 26.65 & 18.1  \\
        & VPP-4  &       &       &       &       \\
        & VPP-10 &       &       &       &       \\
        & Linux  &       &       &       &       \\
\hline
\multirow{4}{*}{\rotatebox{90}{1518B}} &
          VPP-1  & 5.64  & 1.55  &  26.10 & 18.2   \\
        & VPP-4  &       &       &       &       \\
        & VPP-10 &       &       &       &       \\
        & Linux  &       &       &       &       \\
\hline
\end{tabular}
\label{tab:nat-10g}
\end{table}







%----------------------------------
%----------------------------------
%----------------------------------
\subsubsection{25 Gbps Test Results}


\begin{table}[h!]
\centering
\caption{Results of NAT 25~Gbit/s tests}
\begin{tabular}{|c|l|r|r|r|r|}
\hline
\textbf{} & \textbf{Config} & \textbf{Energy [Wh]} & \textbf{Pkt Loss [\%]} & \textbf{Avg Lat [$\mu$s]} & \textbf{Jitter [$\mu$s]} \\
\hline
\multirow{4}{*}{\rotatebox{90}{64B}} &
          VPP-1  & 5.73  & 90.19 & 459.65 & 17.8 \\
        & VPP-4  &       &       &       &       \\
        & VPP-10 &       &       &       &       \\
        & Linux  &       &       &       &       \\
\hline
\multirow{4}{*}{\rotatebox{90}{512B}} &
          VPP-1  & 5.70  & 23.82 & 256.05 & 23.55 \\
        & VPP-4  &       &       &       &       \\
        & VPP-10 &       &       &       &       \\
        & Linux  &       &       &       &       \\
\hline
\multirow{4}{*}{\rotatebox{90}{889B}} &
          VPP-1  & 5.63  & 1.57  & 44.1  & 30.15 \\
        & VPP-4  &       &       &       &       \\
        & VPP-10 &       &       &       &       \\
        & Linux  &       &       &       &       \\
\hline
\multirow{4}{*}{\rotatebox{90}{1280B}} &
          VPP-1  & 5.70  & 1.56  & 35.1  & 15.95 \\
        & VPP-4  &       &       &       &       \\
        & VPP-10 &       &       &       &       \\
        & Linux  &       &       &       &       \\
\hline
\multirow{4}{*}{\rotatebox{90}{1518B}} &
          VPP-1  &  5.81 &  1.56 & 31.2  & 14.4  \\
        & VPP-4  &       &       &       &       \\
        & VPP-10 &       &       &       &       \\
        & Linux  &       &       &       &       \\
\hline
\end{tabular}
\label{tab:nat-25g}
\end{table}






%----------------------------------
%----------------------------------
%----------------------------------
\subsubsection{40 Gbps Test Results}


\begin{table}[h!]
\centering
\caption{Results of NAT 40~Gbit/s tests}
\begin{tabular}{|c|l|r|r|r|r|}
\hline
\textbf{} & \textbf{Config} & \textbf{Energy [Wh]} & \textbf{Pkt Loss [\%]} & \textbf{Avg Lat [$\mu$s]} & \textbf{Jitter [$\mu$s]} \\
\hline
\multirow{4}{*}{\rotatebox{90}{64B}} &
          VPP-1  & 5.78  & 93.89 & 447.95 & 16.6 \\
        & VPP-4  &       &       &       &       \\
        & VPP-10 &       &       &       &       \\
        & Linux  &       &       &       &       \\
\hline
\multirow{4}{*}{\rotatebox{90}{512B}} &
          VPP-1  & 5.79  & 52.02 & 264.75 & 18.45  \\
        & VPP-4  &       &       &       &       \\
        & VPP-10 &       &       &       &       \\
        & Linux  &       &       &       &       \\
\hline
\multirow{4}{*}{\rotatebox{90}{889B}} &
          VPP-1  &  5.82 & 21.93 & 324.95 & 33.5 \\
        & VPP-4  &       &       &       &       \\
        & VPP-10 &       &       &       &       \\
        & Linux  &       &       &       &       \\
\hline
\multirow{4}{*}{\rotatebox{90}{1280B}} &
          VPP-1  & 5.85  & 1.61  & 71.2  & 24.95 \\
        & VPP-4  &       &       &       &       \\
        & VPP-10 &       &       &       &       \\
        & Linux  &       &       &       &       \\
\hline
\multirow{4}{*}{\rotatebox{90}{1518B}} &
          VPP-1  & 5.86  & 1.57  & 48.65 & 26.9  \\
        & VPP-4  &       &       &       &       \\
        & VPP-10 &       &       &       &       \\
        & Linux  &       &       &       &       \\
\hline
\end{tabular}
\label{tab:nat-40g}
\end{table}



%---------------------------------------------------------------

%---------------------------------------------------------------
\section{Presentation and Analysis of Results -> do závěru?}
The evaluation focused on throughput, latency, packet loss, and energy efficiency under different traffic scenarios and system configurations.

As demonstrated by the overall results, the Linux networking stack generally performs better under lighter loads.  
When the packet rate is sufficiently low, the interrupt-based approach proves to be more efficient.
Moreover, under low-traffic conditions, VPP tends to exhibit higher latency, especially when multiple receive queues are configured.  
This may be attributed to VPP’s limited ability to fully leverage the benefits of vectorized packet processing at low volumes.  
When a graph node is triggered -- a relatively costly operation -- incoming packets may have to wait for the current batch to complete.
Enabling Receive Side Scaling in these scenarios can further exacerbate the issue, as limited traffic is distributed across multiple queues and workers, leading to unnecessary overhead.
These observations proved that VPP is not optimized for handling sparse or low-throughput traffic patterns.

On the other hand, when handling higher traffic volumes, VPP processes more fully populated vectors and can take full advantage of vectorization optimizations, 
which helps amortize the per-packet processing cost.
As demonstrated, under heavy load, VPP is able to maintain reasonable latency, while the Linux stack becomes flooded with system calls, 
and its per-packet processing model reveals clear performance limitations.
This is further supported by the observed packet loss in VPP, particularly under heavy NAT workloads with small frame sizes.
It was also observed that VPP’s parallelism become truly apparent only under high-throughput conditions.

A serious drawback of VPP is that it consumes nearly the same amount of energy even when there is no traffic to process.  
While this is less noticeable under typical load conditions, it becomes highly inefficient during periods of minimal or no traffic -- 
such as early morning hours in enterprise or residential networks -- where VPP continues polling the NIC despite having nothing to handle.
VPP does not support on-the-fly hardware configuration changes and typically requires a full restart; a simple reload is not sufficient.  
This further worsens its suitability for dynamic or energy-sensitive environments, where the ability to adjust to changing load conditions -- such as reduced traffic during off-peak hours -- is essential.  
In contrast, the Linux networking stack can accommodate such changes at runtime, allowing CPUs to be enabled, disabled, or reassigned without interrupting network services.

It is important to note that these measurements were conducted using synthetic traffic in a controlled environment, 
designed to evaluate how VPP and the Linux networking stack behave under specific conditions.  
Real-world traffic patterns, including bursts and mixed flow types, may result in different performance characteristics.



%---------------------------------------------------------------


\chapter{Conclusion and Future work}
The main objective of this thesis was to analyze and compare the performance, latency, and energy efficiency of different VPP configurations with a traditional Linux-based router. 
The work aimed to explore the theoretical advantages of VPP, design and implement a custom measurement infrastructure, define relevant traffic scenarios, 
and determine the conditions under which VPP outperforms the Linux network stack.

The theoretical part of the thesis provided a detailed explanation of VPP's architecture and operating principles. 
It also included a comparative analysis of packet processing between VPP and the traditional Linux network stack, highlighting the key differences in design, performance potential, and resource utilization.

As part of the practical work, a custom measurement infrastructure was designed and implemented to enable reliable and repeatable testing. 
Using this setup, a total of 4800 measurements, reflecting realistic usage conditions on standard COTS hardware, 
were conducted across 60 different test scenarios using four configurations -- three VPP variants and one Linux-based reference system.
This resulted in 240 distinct test cases, providing a robust dataset for evaluating throughput, latency, and energy efficiency under a variety of traffic conditions.

The results show that VPP consistently outperformed the Linux-based reference system in terms of throughput and latency -- with the exception of low-traffic forwarding scenarios with higher frame sizes, 
where the Linux stack achieved slightly better results. 
However, these differences were relatively minor. 
In terms of energy efficiency, the Linux-based reference system was generally more efficient than all VPP configurations, with the exception of VPP-1, which outperformed Linux in all cases.
VPP also demonstrated more stable performance.
Given VPP’s overall higher performance ceiling, it is reasonable to expect that it would handle sudden traffic spikes more effectively and with greater stability compared to the Linux-based solution. 
The results also clearly indicate that VPP would be more resilient under DoS or DDoS-style attack conditions.


Several directions for future work remain open. For small-to-mid-sized ISPs, it would be useful to compare VPP not only with the Linux network stack, but 
also with commercial or legacy routers commonly used in practice. 
It would also be valuable to evaluate the impact of advanced features such as Quality of Service (QoS), firewalling, or traffic shaping. 
Finally, comparing VPP with a custom-tuned Linux kernel optimized for networking could provide deeper insight into the limits of traditional software-based routing.
