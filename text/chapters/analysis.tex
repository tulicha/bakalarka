The evaluation focused on throughput, latency, packet loss, and energy efficiency under different traffic scenarios and system configurations.

As demonstrated by the overall results, the Linux networking stack generally performs better under lighter loads.  
When the packet rate is sufficiently low, the interrupt-based approach proves to be more efficient.
Moreover, under low-traffic conditions, VPP tends to exhibit higher latency, especially when multiple receive queues are configured.  
This may be attributed to VPP’s limited ability to fully leverage the benefits of vectorized packet processing at low volumes.  
When a graph node is triggered -- a relatively costly operation -- incoming packets may have to wait for the current batch to complete.
Enabling Receive Side Scaling in these scenarios can further exacerbate the issue, as limited traffic is distributed across multiple queues and workers, leading to unnecessary overhead.
These observations proved that VPP is not optimized for handling sparse or low-throughput traffic patterns.

On the other hand, when handling higher traffic volumes, VPP processes more fully populated vectors and can take full advantage of vectorization optimizations, 
which helps amortize the per-packet processing cost.
As demonstrated, under heavy load, VPP is able to maintain reasonable latency, while the Linux stack becomes flooded with system calls, 
and its per-packet processing model reveals clear performance limitations.
This is further supported by the observed packet loss in VPP, particularly under heavy NAT workloads with small frame sizes.
It was also observed that VPP’s parallelism become truly apparent only under high-throughput conditions.

A serious drawback of VPP is that it consumes nearly the same amount of energy even when there is no traffic to process.  
While this is less noticeable under typical load conditions, it becomes highly inefficient during periods of minimal or no traffic -- 
such as early morning hours in enterprise or residential networks -- where VPP continues polling the NIC despite having nothing to handle.
VPP does not support on-the-fly hardware configuration changes and typically requires a full restart; a simple reload is not sufficient.  
This further worsens its suitability for dynamic or energy-sensitive environments, where the ability to adjust to changing load conditions -- such as reduced traffic during off-peak hours -- is essential.  
In contrast, the Linux networking stack can accommodate such changes at runtime, allowing CPUs to be enabled, disabled, or reassigned without interrupting network services.

It is important to note that these measurements were conducted using synthetic traffic in a controlled environment, 
designed to evaluate how VPP and the Linux networking stack behave under specific conditions.  
Real-world traffic patterns, including bursts and mixed flow types, may result in different performance characteristics.


