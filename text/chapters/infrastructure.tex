The testing infrastructure has been implemented as recomended in RFC 2544~\cite{rfc2544}, which defines methods for evaluating network performance. 
It consists of a device under test (DUT), connected to a measurement device called Tester\footnote{The hardware used in this testing setup was loaned free of charge for the purposes of this bachelor thesis by Silicon Hill club.}.

The Device Under Test (DUT) and the measurement device are connected using 100~Gbit capable cables, preventing any potential bottlenecks in the connection.
The illustration of this hardware setup is shown in fig. \ref{fig:hardware-setup}

\begin{figure}[!htbp]
    \centering
    \includegraphics[width=0.9\linewidth]{images/setup.png}
    \caption{Picture showing hardware setup}
    \label{fig:hardware-setup}
\end{figure}

The Device Under Test (DUT) is the network device being evaluated during testing. 
It is configured with a specific network stack and settings based on measurement scenario 
and serves as the focus of performance and behavior analysis in a controlled test environment. 
The DUT is responsible for processing network traffic and responding to the test conditions set by the measurement device.
Additionally, the electrical power consumption of the DUT is monitored and measured during the tests to assess its energy efficiency under varying loads.
The hardware of DUT is shown in table \ref{tab:hardware_dut}.

\begin{table}[h!]
\centering
\caption{Hardware details for DUT (Device Under Test)}
\begin{tabular}{|c|c|}
\hline
\textbf{Hardware Component} & \textbf{DUT (Device Under Test)} \\
\hline
\textbf{CPU Model} & 2x Intel(R) Xeon(R) CPU E5-2660 v3 \\
\hline
\textbf{Frequency} & 2.60GHz \\
\hline
\textbf{Cores} & 10 physical cores each (one thread per core) \\
\hline
\textbf{Memory (RAM)} & TODO! Size, type, speed \\
\hline
\textbf{Network Interface Cards (NIC)} & Mellanox ConnectX-6 Dx (Dual-port) \\
\hline
\end{tabular}
\label{tab:hardware_dut}
\end{table}

The Tester (Measurement Device), on the other hand, is responsible in generating the network traffic and capturing the responses from the DUT.
Its physical features are shown in table \ref{tab:hardware_tester}. 

\begin{table}[h!]
\centering
\caption{Hardware details for Tester (Measurement Device)}
\begin{tabular}{|c|c|}
\hline
\textbf{Hardware Component} & \textbf{Tester (Measurement Device)} \\
\hline
\textbf{CPU Model} & 2x Intel(R) Xeon(R) Gold 6136 CPU \\
\hline
\textbf{Frequency} & 3.00GHz \\
\hline
\textbf{Cores} & 12 physical cores each (two threads per core)\\
\hline
\textbf{Memory (RAM)} & TODO! Size, type, speed \\
\hline
\textbf{Network Interface Cards (NIC)} & 2x Mellanox ConnectX-5 \\
\hline
\end{tabular}
\label{tab:hardware_tester}
\end{table}

The DUT is running Debian GNU/Linux 12 (Bookworm) x86\_64 with Linux kernel version \textit{6.1.0-32-amd64}, VPP v25.02-release, and DPDK version 24.11.1. 
This kernel version is the current standard long-term support (LTS) release provided with Debian 12 (Bookworm) and was used for all tests involving the Linux networking stack.

The tester generates traffic using Cisco TRex version 3.06.
