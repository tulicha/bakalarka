\subsection{DPDK and Its Role in VPP}

The Data Plane Development Kit (DPDK) is an open-source collection of libraries and drivers designed to support high-speed packet processing in user space. 
It was initially developed by Intel in 2010 and is now maintained as a Linux Foundation project. 
DPDK provides a set of APIs and components that allow applications to bypass the kernel network stack and directly access network interface cards (NICs) 
through poll-mode drivers, significantly reducing the overhead associated with traditional packet handling mechanisms.\cite{dpdk_about}

The DPDK completely bypasses the kernel, communicating directly with the NIC.
DPDK avoids the use of the kernel’s system calls, instead handling its own I/O synchronization and memory management. 
DPDK employs a Poll Mode Driver (PMD) that uses busy-polling to retrieve, process, and deliver network packets to user-space applications without relying on interrupts. 
While this approach enhances performance by reducing latency, it also results in high CPU utilization, with the CPU usage on each core remaining close to 100\% regardless of the network load.\cite{FREITAS2022148}

DPDK is used in VPP for interfacing with hardware. It is implemented as a plugin called \textit{dpdk-plugin}.\cite{LINGUAGLOSSA, DR:COMMAG-18} 

\subsection{Architecture of VPP Modules}
\subsection{Configuration and Startup}
\subsection{Plugins and Extensibility}
