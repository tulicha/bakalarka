\subsection{DPDK and Its Role in VPP}

přesunout

The Data Plane Development Kit (DPDK) is an open-source collection of libraries and drivers designed to support high-speed packet processing in user space. 
It was initially developed by Intel in 2010 and is now maintained as a Linux Foundation project. 
DPDK provides a set of APIs and components that allow applications to bypass the kernel network stack and directly access network interface cards (NICs) 
through poll-mode drivers, significantly reducing the overhead associated with traditional packet handling mechanisms.\cite{dpdk_about}

The DPDK completely bypasses the kernel, communicating directly with the NIC.
DPDK avoids the use of the kernel’s system calls, instead handling its own I/O synchronization and memory management. 
DPDK employs a Poll Mode Driver (PMD) that uses busy-polling to retrieve, process, and deliver network packets to user-space applications without relying on interrupts. 
While this approach enhances performance by reducing latency, it also results in high CPU utilization, with the CPU usage on each core remaining close to 100\% regardless of the network load.\cite{FREITAS2022148}

DPDK is used in VPP for interfacing with hardware. It is implemented as a plugin called \textit{dpdk-plugin}.\cite{LINGUAGLOSSA, DR:COMMAG-18} 

%------------------------------------------
\subsection{VPP key architecture components}

VPP's dataplane is implemented by four main architectural layers: VPPINFRA, VNET, VLIB, and Plugins. 
Each layer provides distinct functionalities that support efficient networking operations, from low-level data structure management to high-level network function optimizations. 
The following sections describe these layers in detail: \footnote{https://my-vpp-docs.readthedocs.io/en/vpp-config/gettingstarted/developers/swarch/softwarearchitecture.html}

\begin{itemize}
  \item \textbf{VPPINFRA} -- layer providing foundational libraries for performing tasks with memory, vectors, rings, lookups in hash tables \& timers.
  \item \textbf{VNET} -- layer that deals with networking on layers 2 - 4 and is responsible for Control plane.
  \item \textbf{VLIB} -- layer that provides library for vector processing, implements CLI and handles application management functions.
  \item \textbf{Plugins} -- layer which is a set of plugins that allow for adding network functions and optimizations tailored to specific needs.
\end{itemize}

\subsubsection{VPPINFRA}
VPPINFRA is a collection of library services designed to offer high-performance capabilities for various tasks. 
It includes features such as dynamic arrays, hashes, bitmaps, high-precision real-time clock support, event logging, and data structure serialization. The following functionalities are implemented:

\begin{itemize}
  \item \textbf{Vectors} -- dynamically resized arrays with \textit{headers} defined by user. They serve as a core building block for other data structures (e.g., hash tables, pools) and allow efficient memory reuse via safe length resetting.
  \item \textbf{Bitmaps} -- dynamic bitmaps based on VPPINFRA vectors.
  \item \textbf{Pools} -- structures used to quickly allocate \& free fixed-size data structures. 
  \item \textbf{Hashes} -- structures thats provide fast key-value lookups, commonly mapping keys to indices in vectors or pools. Bihash is used in the data plane for fixed-size keys and is thread-safe, while the simpler hash is used in the control plane for exact string matching.
  \item \textbf{Timekeeping} -- service providing high-precision, low-cost timing based on CPU ticks. Since CPU ticks are not perfectly accurate, the system continuously adjusts its estimate of "ticks per second" by comparing with the kernel’s time. This results in precise and smooth time measurements without the need for expensive system calls. 
  \item \textbf{Timer wheel} -- system for efficiently managing timers or timeouts. It allows the user to define parameters like the number of wheels, slots per ring, and timers per object, optimizing time-based operations in systems requiring high-performance event management.
\end{itemize}

\subsubsection{VNET}
odmítám dělat teď

\subsubsection{VLIB}
Zítra je taky den

\subsubsection{Plugins}
Plugins are used to modify or create new features into the VPP. 
Developers can create plugins through a straightforward process, involving the generation of necessary files and integration into the system. 
After building, the new plugin can be loaded and tested within the VPP environment. 

VLIB supports a simple mechanism for loading and using plugins. 
VLIB client applications specify a directory where the plug-ins are located and can apply a filter to narrow down the search. 
Once the plug-ins are loaded, VLIB ensures they are correctly registered and ready for use.


%--------------------------------------
\subsection{Configuration and Startup}
