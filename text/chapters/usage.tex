VPP supports a comprehensive set of Layer 2 to Layer 4 network functions. 
At Layer 2, it provides Ethernet bridging, MAC learning, VLAN tagging (including dot1q and QinQ), and support for L2 cross-connects and policers.

At Layer 3, VPP implements both IPv4 and IPv6 routing with ECMP support, NAT44/NAT64, and ACL-based filtering. 
It also supports tunneling mechanisms such as GTP-U, IP-in-IP, and VXLAN. Segment routing (SRv6), LISP, and punt redirect mechanisms are included as well.

At the transport layer (L4), basic UDP and TCP stack functionality is available. 

Additionally, supported features include PPPoE, the WireGuard VPN protocol, GRE tunneling, DHCP client and proxy functionality, and L2TPv3.\cite{fdio-vpp-features-2502}

According to the VPP's authors~\cite{fdio_what_is_vpp}, VPP can be for example effectively utilized as a virtual switch, virtual router, gateway or used as a basis for a firewall, IDS and load balancer.
It already includes enough features to be deployed in production environments.


\subsection{Integration with the SDN/NFV Ecosystem}
To meet the requirements of modern virtualized and cloud-native networking environments, Vector Packet Processing (VPP) was architected with a clear separation between the data plane and control plane. 
This design choice enables its integration into SDN and NFV frameworks, where packet forwarding logic can operate independently from centralized control mechanisms. 
VPP's modularity and userspace implementation allow it to function efficiently within dynamic, multi-tenant infrastructure, 
while remaining compatible with orchestration systems and control-plane protocols commonly used in such deployments

VPP is fully compatible with both Virtual Network Functions (VNFs) and Cloud-Native Network Functions (CNFs). 
Its modular architecture allows deployment in environments utilizing service function chaining, Kubernetes-based orchestration, or OpenStack-based infrastructures. 
Because of its userspace design and performance-optimized data plane, VPP can serve as the fast packet processing backend for SDN-controlled systems and NFV orchestrators.\cite{fdio2017whitepaper}

\subsection{VPP as a Complete Router Solution}
VPP is implemented solely as a data-plane, meaning it is not a complete routing solution on its own. 
VPP is dedicated to efficiently forwarding packets between interfaces based on routing rules and access control filters, 
but it does not include a native control-plane or support for dynamic routing protocols such as BGP or OSPF.

However, as demonstrated by the authors of the VBSR (VPP-Bird Software Router) project \cite{10819057}, 
it is possible to integrate VPP with additional components such as the Linux Control Plane (Linux-CP) plugin and the BIRD routing daemon. 
Bird acting as a control-plane enables dynamic routing using protocols like BGP 
and the Linux-CP is responsible for communication between VPP and BIRD. 
This integrated system creates a nearly feature-complete router solution, comparable in functionality to commercial routers.

It is important to note, however, that firewall functionality is still limited and was left by authors of VBSR as a future work.~\cite{10819057} 
While VPP supports basic packet filtering through ACLs, it lacks advanced stateful firewall features~\cite{fdio-vpp-features-2502}. These would need to be handled externally.

