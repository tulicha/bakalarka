
\subsection{VPP as a Complete Router Solution}
Vector Packet Processing (VPP) is implemented solely as a data-plane, meaning it is not a complete routing solution on its own. 
VPP is dedicated to efficiently forwarding packets between interfaces based on routing rules and access control filters, 
but it does not include a native control-plane or support for dynamic routing protocols such as BGP or OSPF.

However, as demonstrated by the authors of the VBSR (VPP-Bird Software Router) project \cite{10819057}, 
it is possible to integrate VPP with additional components such as the Linux Control Plane (Linux-CP) plugin and the BIRD routing daemon. 
Bird acting as a control-plane enables dynamic routing using protocols like BGP 
and the Linux-CP is responsible for communication between VPP and BIRD 
This integrated system creates a nearly feature-complete router solution, comparable in functionality to commercial routers.

It is important to note, however, that firewall functionality is still limited and was left by authors of VBSR as a future work.\cite{10819057} 
While VPP supports basic packet filtering through ACLs, it lacks advanced stateful firewall features\cite{fdio-vpp-features-2502}. These would need to be handled externally.

