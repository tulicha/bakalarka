The RFC 2544 recommends that each test be at least 60 seconds in duration~\cite{rfc2544}. 
In this work, the duration was extended to 120 seconds and each test was repeated 20 times to ensure greater stability and statistical relevance of the results. 
The reported values represent the arithmetic mean of these 20 measurements.
In cases where an error, anomalous spike or irregularity was observed in the results, the corresponding measurement was discarded and the test was repeated. 
All these steps were taken to ensure consistency and statistical reliability.

The following metrics were collected: the number of transmitted and received packets and bytes; average, minimum, and maximum one-way latency; jitter; 
and total energy consumption of the DUT, expressed in watt-hours.

All numerical results are rounded to two decimal places.
Since traffic generation was performed using TRex, its potential measurement inaccuracy must be taken into account when interpreting the results.
It should also be noted that in cases where packet loss occurred, higher deviations in delay and jitter statistics can be expected, as packet loss may negatively impact the consistency of these measurements.
To evaluate energy efficiency, the number of packets per watt-hour (PPWh) and bytes per watt-hour (BPWh) was used, considering only successfully delivered packets.
Power consumption was measured using a Raritan PX3-5498-K1 unit running firmware version 4.2.0.5-50274.
The idle power consumption of the DUT is approximately 5 Wh per two minutes.
