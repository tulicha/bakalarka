\subsection{NAT}
Network Address Translation (NAT) is a commonly used technique among small to mid-sized ISPs for conserving public IP addresses and managing internal networks. 
Although NAT is a standard practice, it introduces additional complexity into packet forwarding.
Specifically, NAT requires the router to maintain stateful information for each active connection. 
For every translated flow, the router must store the original and translated source IP addresses and ports, along with the destination address and protocol type. 
This information is necessary to correctly map incoming response packets back to the appropriate internal hosts.
Since NAT modifies headers -- such as the source IP addresses and source ports -- all affected checksums must be recalculated, which imposes further computational overhead.
The goal of this test is to evaluate how NAT impacts overall system performance.

For this purpose, source NAT is applied on top of the one-way forwarding scenario, meaning that the test setup is identical to the basic forwarding case, but with NAT translation enabled at the DUT. 
A /16 source IP address pool is used to emulate a large number of concurrent clients, while the Device Under Test (DUT) has a /24 pool of inside global IP addresses available for translation. 
This configuration serves as a stress test for the NAT implementation and its capacity to manage many concurrent state entries.
NAT was implemented using \textit{nftables} in Linux, and the \textit{nat\_plugin} in VPP.

A one-way traffic pattern is used to isolate the performance impact of NAT translations without interference from return traffic, 
as response packets would be subject to the same translation and state tracking processes, merely in the opposite direction.

%----------------------------------
%----------------------------------
%----------------------------------
\subsubsection{1 Gbps Test Results}

As the results in Table~\ref{tab:nat-1g} show, the introduction of NAT had a slight impact on all VPP configurations compared to plain one-way forwarding at 1~Gbps.  
All VPP configurations also exhibited a consistent packet loss of approximately 1.5\%.
Since this behavior is present across all test cases, it suggests that VPP may drop the first few packets of each flow until a NAT session table entry is established.  
The stateless UDP traffic generated by TRex provides very limited context for session tracking, which likely contributes to this issue.  
In real-world network traffic -- typically bidirectional and stateful -- this behavior would likely not occur or would be mitigated by application-layer retries.
Interestingly, Linux delivered all packets with low latency except in the 64-byte frame test.  
The slightly better results of Linux with NAT compared to plain forwarding may be explained by the presence of precomputed conntrack sessions, which can accelerate routing decisions in the kernel.

\begin{table}[h!]
\centering
\caption{Results of NAT 1~Gbit/s tests}
\begin{tabular}{|c|l|r|r|r|r|}
\hline
\textbf{} & \textbf{Config} & \textbf{Energy [Wh]} & \textbf{Pkt Loss [\%]} & \textbf{Avg Lat [$\mu$s]} & \textbf{Jitter [$\mu$s]} \\
\hline
\multirow{4}{*}{\rotatebox{90}{64B}} &
          VPP-1  & 5.67  & 1.56  & 19.10 & 11.65 \\
        & VPP-4  & 6.37  & 1.56  & 35.75 & 16.35 \\
        & VPP-10 & TODO? &       &       &       \\
        & Linux  & 6.61  & 4.43  & 303.25 & 182.85 \\
\hline
\multirow{4}{*}{\rotatebox{90}{512B}} &
          VPP-1  & 5.76  &  1.53 & 12.25  & 12.25 \\
        & VPP-4  & 6.36  &  1.53 & 26.60  & 19.30 \\
        & VPP-10 & 7.90  &  1.53 & 25.60  & 18.25 \\
        & Linux  & 6.21  &  0.00 & 18.50  & 10.25 \\
\hline
\multirow{4}{*}{\rotatebox{90}{889B}} &
          VPP-1  & 5.74  & 1.50  & 10.30  & 7.45   \\
        & VPP-4  & 6.26  & 1.50  & 23.40  & 18.40  \\
        & VPP-10 & 7.81  & 1.51  & 23.80  & 20.10  \\
        & Linux  & 6.13  & 0.00  & 12.30  & 11.30  \\
\hline
\multirow{4}{*}{\rotatebox{90}{1280B}} &
          VPP-1  & 5.73  & 1.48  & 7.70  &  7.30 \\
        & VPP-4  & 6.35  & 1.48  & 18.45 & 15.45 \\
        & VPP-10 & TODO? &       &       &       \\
        & Linux  & 6.07  & 0.00  & 11.4  & 1.3   \\
\hline
\multirow{4}{*}{\rotatebox{90}{1518B}} &
          VPP-1  & 5.68  & 1.47  & 7.30  & 5.55  \\
        & VPP-4  & 6.35  & 1.47  & 17.55 & 16.5  \\
        & VPP-10 & TODO? &       &       &       \\
        & Linux  & 6.05  & 0.00  & 10.55 & 6.60  \\
\hline
\end{tabular}
\label{tab:nat-1g}
\end{table}


%----------------------------------
%----------------------------------
%----------------------------------
\subsubsection{10 Gbps Test Results}

In this higher-traffic scenario, Table~\ref{tab:nat-10g} shows that the increased packet rate had a noticeable impact on Linux, 
resulting in significantly worse performance compared to plain 10~Gbps one-way forwarding.
This highlights the substantial computational overhead introduced by NAT on top of forwarding, particularly at high packet-per-second rates.
The results of the 64-byte frame test reveal a performance degradation in VPP compared to forwarding alone.  
However, aside from this specific case, VPP consistently delivered only slightly worse -- but still comparable -- results to plain forwarding.
In this test, VPP's superior packet processing efficiency compared to Linux becomes clearly visible.

\begin{table}[h!]
\centering
\caption{Results of NAT 10~Gbit/s tests}
\begin{tabular}{|c|l|r|r|r|r|}
\hline
\textbf{} & \textbf{Config} & \textbf{Energy [Wh]} & \textbf{Pkt Loss [\%]} & \textbf{Avg Lat [$\mu$s]} & \textbf{Jitter [$\mu$s]} \\
\hline
\multirow{4}{*}{\rotatebox{90}{64B}} &
          VPP-1  & 5.75  & 75.12 & 259.15 & 11.95 \\
        & VPP-4  & 6.17  & 51.98 & 735.00 & 55.20 \\
        & VPP-10 & 7.95  & 5.86  & 676.15 & 76.15 \\
        & Linux  & 7.25  & 84.03 & 10314.40 & 1683.40 \\
\hline
\multirow{4}{*}{\rotatebox{90}{512B}} &
          VPP-1  & 5.56  & 1.56  & 31.15 & 16.90 \\
        & VPP-4  & 6.20  & 1.56  & 40.10 & 17.45 \\
        & VPP-10 & 7.76  & 1.56  & 32.40 & 17.65 \\
        & Linux  & 6.87  & 9.02  & 255.10 & 177.45  \\
\hline
\multirow{4}{*}{\rotatebox{90}{889B}} &
          VPP-1  & 5.57  & 1.56  & 29.80 & 18.55 \\
        & VPP-4  & 6.28  & 1.56  & 34.45 & 21.60 \\
        & VPP-10 & 7.78  & 1.56  & 29.45 & 17.00 \\
        & Linux  & 6.63  & 0.00  & 166.20 & 129.60 \\
\hline
\multirow{4}{*}{\rotatebox{90}{1280B}} &
          VPP-1  & 5.66  & 1.55  & 26.65 & 18.10 \\
        & VPP-4  & 6.26  & 1.55  & 31.50 & 19.50 \\
        & VPP-10 & 7.79  & 1.56  & 28.55 & 17.20 \\
        & Linux  & 6.49  & 0.00  & 101.20 & 93.40  \\
\hline
\multirow{4}{*}{\rotatebox{90}{1518B}} &
          VPP-1  & 5.64  & 1.55  &  26.10 & 18.20  \\
        & VPP-4  & 6.21  & 1.55  &  30.15 & 19.35  \\
        & VPP-10 & 7.88  & 1.55  &  26.55 & 18.60  \\
        & Linux  & 6.45  & 0.00  &  67.25 & 61.60  \\
\hline
\end{tabular}
\label{tab:nat-10g}
\end{table}

The energy efficiency graph shown in Fig.~\ref{fig:nat-10g} demonstrates results similar to those observed in plain one-way 10\,Gbps forwarding.

\begin{figure}[!htbp]
    \centering
    \includegraphics[width=\linewidth]{images/consumption-nat-10g.png}
    \caption{Energy efficiency per delivered data in NAT 10\,Gbit/s.}
    \label{fig:nat-10g}
\end{figure}

%----------------------------------
%----------------------------------
%----------------------------------
\subsubsection{25 Gbps Test Results}

As the results presented in Tab.~\ref{tab:nat-25g} demonstrate, in the 64-byte frame scenario even the VPP-10 configuration dropped more than half of all packets. 
These results are more comparable to the 40~Gbit/s basic forwarding scenario.
In the remaining tests, all VPP latency metrics approximately doubled compared to 25~Gbit/s basic forwarding. The Linux configuration struggled with high latency and severe packet loss.
Overall, this test further demonstrates VPP's ability to handle high-throughput traffic, although the NAT processing significantly increases computational load.

\begin{table}[h!]
\centering
\caption{Results of NAT 25~Gbit/s tests}
\begin{tabular}{|c|l|r|r|r|r|}
\hline
\textbf{} & \textbf{Config} & \textbf{Energy [Wh]} & \textbf{Pkt Loss [\%]} & \textbf{Avg Lat [$\mu$s]} & \textbf{Jitter [$\mu$s]} \\
\hline
\multirow{4}{*}{\rotatebox{90}{64B}} &
          VPP-1  & 5.73  & 90.19 & 459.65 & 17.80 \\
        & VPP-4  & TODO? &       &       &        \\
        & VPP-10 & 7.95  & 59.86 & 851.35 & 57.60 \\
        & Linux  & 7.26  & 93.62 & 8398.85 & 1636.1 \\
\hline
\multirow{4}{*}{\rotatebox{90}{512B}} &
          VPP-1  & 5.70  & 23.82 & 256.05 & 23.55 \\
        & VPP-4  & 6.50  & 1.56  & 71.55 & 23.05  \\
        & VPP-10 & 7.95  & 1.56  & 40.05 & 15.85  \\
        & Linux  & 7.30  & 51.35 & 4566.95 & 408.6 \\
\hline
\multirow{4}{*}{\rotatebox{90}{889B}} &
          VPP-1  & 5.63  & 1.57  & 44.10 & 30.15 \\
        & VPP-4  & 6.61  & 1.56  & 47.45 & 21.40 \\
        & VPP-10 & 7.91  & 1.56  & 35.70 & 21.30 \\
        & Linux  & 7.02  & 28.91 & 6797.05 & 403.45 \\
\hline
\multirow{4}{*}{\rotatebox{90}{1280B}} &
          VPP-1  & 5.70  & 1.56  & 35.10 & 15.95 \\
        & VPP-4  & 6.63  & 1.56  & 43.05 & 21.60 \\
        & VPP-10 & 7.95  & 1.56  & 36.70 & 18.20 \\
        & Linux  & 6.92  & 12.54 & 262.25 & 204.50  \\
\hline
\multirow{4}{*}{\rotatebox{90}{1518B}} &
          VPP-1  &  5.81 &  1.56 & 31.20 & 14.40 \\
        & VPP-4  &  6.63 & 1.56  & 40.65 & 22.50 \\
        & VPP-10 &  8.03 & 1.56  & 34.40 & 18.45 \\
        & Linux  &  TBD   &       &       &       \\
\hline
\end{tabular}
\label{tab:nat-25g}
\end{table}






%----------------------------------
%----------------------------------
%----------------------------------
\subsubsection{40 Gbps Test Results}


\begin{table}[h!]
\centering
\caption{Results of NAT 40~Gbit/s tests}
\begin{tabular}{|c|l|r|r|r|r|}
\hline
\textbf{} & \textbf{Config} & \textbf{Energy [Wh]} & \textbf{Pkt Loss [\%]} & \textbf{Avg Lat [$\mu$s]} & \textbf{Jitter [$\mu$s]} \\
\hline
\multirow{4}{*}{\rotatebox{90}{64B}} &
          VPP-1  & 5.78  & 93.89 & 447.95 & 16.60 \\
        & VPP-4  & 6.41  & 87.59 & 830.80 & 44.80 \\
        & VPP-10 & 7.98  & 74.47 & 860.45 & 55.85 \\
        & Linux  &       &       &        &       \\
\hline
\multirow{4}{*}{\rotatebox{90}{512B}} &
          VPP-1  & 5.79  & 52.02 & 264.75 & 18.45  \\
        & VPP-4  & 6.61  & 17.80 & 793.65 & 64.10  \\
        & VPP-10 & 8.00  & 1.56  & 49.75  & 17.15  \\
        & Linux  &       &       &       &       \\
\hline
\multirow{4}{*}{\rotatebox{90}{889B}} &
          VPP-1  &  5.82 & 21.93 & 324.95 & 33.50 \\
        & VPP-4  &  6.62 & 1.56  & 65.25  & 22.60 \\
        & VPP-10 &  8.05 & 1.56  & 44.10  & 17.90  \\
        & Linux  &       &       &       &       \\
\hline
\multirow{4}{*}{\rotatebox{90}{1280B}} &
          VPP-1  & 5.85  & 1.61  & 71.20 & 24.95 \\
        & VPP-4  & 6.59  & 1.56  & 51.60 & 21.25 \\
        & VPP-10 & 7.98  & 1.56  & 41.60 & 19.95 \\
        & Linux  &       &       &       &       \\
\hline
\multirow{4}{*}{\rotatebox{90}{1518B}} &
          VPP-1  & 5.86  & 1.57  & 48.65 & 26.90 \\
        & VPP-4  & 6.59  & 1.56  & 47.80 & 20.60 \\
        & VPP-10 & 8.10  & 1.56  & 39.00 & 19.45 \\
        & Linux  &       &       &       &       \\
\hline
\end{tabular}
\label{tab:nat-40g}
\end{table}

